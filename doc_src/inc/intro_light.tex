\chapter{Informations}

Cette brochure est la version «light» du Passeport vacances Fribourg 2016. Comme le délai est passé pour les
activités avec inscription, cette brochure ne décrit que les activités qui ne nécessitent pas d'inscription.

\section*{Organisateurs}

Passeport vacances Fribourg, ch. du Grabensaal 4, 1700 Fribourg

\section*{Site Internet}

\url{www.passeport-vacances-fribourg.ch}

\section*{Facebook}

\url{https://www.facebook.com/passeportvacancesfribourg}

\section*{Adresse électronique du secrétariat}

\url{passvac.fribourg@bluewin.ch}

\section*{Age}

Le passeport est destiné aux enfants et aux jeunes de 7 ans révolus à 16 ans.

\section*{Prix et durée du passeport}

Le prix d'un passeport est de 35 francs (sauf dans les rares cas où la commune du détenteur ne participe pas financièrement). 

Il est valable deux semaines au choix entre le 11 juillet et le 7 août 2016 et permet de participer à la grande fête finale qui aura lieu le \textbf{vendredi 5 août 2016, même avec un passeport périmé}. 

Le passeport du troisième enfant d'une même famille est gratuit, sur présentation du livret de famille. Dans ce cas, les passeports doivent être achetés en même temps mais les dates d'utilisation peuvent être différentes. 

Le passeport est personnel et les enfants doivent l'avoir sur eux lors de chaque activité. En cas de perte ou s'il n'est pas utilisé, il n'est pas remplacé, ni remboursé. 

\section*{Prestations (v. "Activités spontanées et à coupon")}

Dans le prix du passeport sont compris:

\begin{itemize}
\item la fête de clôture qui aura lieu le vendredi 5 août 2016, même avec un passeport périmé;
\item un libre accès à l'ensemble du réseau régional et urbain des TPF;
\item l'activité "L'atelier des petits gourmands" qui aura lieu les 2, 3, 4 et 6 août à Fribourg Centre (inscriptions sur www.fribourg-centre.com)
\item divers coupons à utiliser pour une partie gratuite de minigolf ou pour se rendre dans des commerces de la place qui vous offriront une boisson, une viennoiserie ou autres;
\item l'entrée gratuite dans plusieurs musées, à la piscine du Levant et aux Bains de la Motta;
\item l'accès gratuit à la Tour de la Cathédrale Saint-Nicolas; 
\item le prêt de livres et jeux;
\item un tour en petit train du Gottéron;
\item du hockey et du patinage libre les samedis 6, 13, 20 et 27 août;
\item le concours "le Carnaval des animaux de Camille Saint-Saëns";
\item l'accueil midi, tous les jours, du lundi au vendredi de 11h30 à 13h30, à la ferme du Grabensaal (prendre un pique-nique avec boisson). A certaines dates, les jeunes peuvent également amener des grillades (v. l'activité "Grillades au Grabensaal").
\end{itemize}

\section*{Déroulement}

\subsection*{Activités avec inscription}

Le délai d'inscription pour les activités avec inscription est dépassé et ces activités ne sont plus disponibles.


\section*{Points de vente des passeports}

\begin{itemize}
\item Office du tourisme de Fribourg, place Jean-Tinguely 1 (bâtiment Equilibre)
\end{itemize}

\section*{Bien des activités ayant lieu dans les endroits suivants, voici comment vous y rendre:}

\subsection*{La ferme du Grabensaal}

Prenez le bus TPF ligne 4 Auge, arrêt "Sous-Pont". Puis, en face de l'école des Neigles, traversez la passerelle qui mène à la ferme;

\subsection*{Gottéron}

Les locaux du Gottéron se trouvent au ch. du Gottéron 15 et 17. Prenez le bus TPF ligne 4 Auge, jusqu'à l'arrêt "La Palme". Dirigez-vous ensuite vers la vallée du Gottéron et marchez environ 7 minutes. 

\section*{Horaires}

Nous demandons aux participants d'être à l'heure. Les organisateurs ou les accompagnants ne sont pas tenus d'attendre. Lorsque le bus est utilisé pour se rendre à une activité, il se peut que l'heure du retour ne soit pas tout à fait respectée, en raison du trafic sur les routes. Nous vous remercions d'avance pour votre compréhension.

\section*{Discipline}

Les personnes inscrites à une activité s'engagent à y participer. Après chaque activité, les jeunes remercient l'organisateur.

\section*{Assurances}

Tous les participants doivent être assurés en cas d'accident et sont responsables des dommages qu'ils pourraient causer. Les organisateurs déclinent toute responsabilité en cas d'accident.

\section*{Comment nous soutenir ?}

\begin{description}
	\item [Par un don]
	Chaque don sera apprécié à sa juste valeur (CCP 17-1741-4)
	\item [En nous proposant du matériel]
	Nous avons toujours besoin de matériel divers pour réaliser et préparer
	nos activités. Vous pouvez nous faire des propositions en nous adressant
	un courriel à\\
	\url{pass\_vac@hotmail.fr}
	\item [Par du bénévolat]
	Si vous souhaitez participer bénévolement à nos activités pour les
	enfants, n’hésitez pas à nous envoyer un courriel à \url{pass_vac@hotmail.fr}
\end{description}

\textbf{Les organisateurs remercient toutes les personnes qui, par leur aide bénévole et leur disponibilité, permettent la mise sur pied du Passeport vacances. Un merci tout particulier à tous les généreux donateurs pour leur appui financier.
}
